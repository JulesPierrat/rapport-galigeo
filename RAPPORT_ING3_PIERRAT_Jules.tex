%----------------------------------------------------------------
% FEUILLE DE STYLE ENSG au format Latex
%Classe de document pour le thème Latex de l'ENSG
% v.1 sept. 2010, D Lercier : création
% v.2 sept. 2012, T Coupin : création fichier classe et modif.
% v.3 sept. 2016, J. Beilin, modification de la gestion de la biblio + modifs mineures 
%----------------------------------------------------------------

\documentclass{themeensg}
% \usepackage{fontspec}
\usepackage{color}
\usepackage{graphicx}
\usepackage{float}
\usepackage{caption}

%---Texte en filigranne---
\SetWatermarkText{\textsc{Brouillon}}
%pour l'enlever : \SetWatermarkText{}
\SetWatermarkText{}

% mise en page no-stress
\renewcommand{\familydefault}{\sfdefault}
%-------------------------

%---Mes packages à moi---
%\usepackage{}
%------------------------

%---Mes raccourcis---
\newcommand{\transpose}[1]{{}^t \! #1}
\newcommand{\ensg}{\textsc{Ensg}}

\renewcommand{\author}{Jules Pierrat}

%--------------------

%---Paramètres du pdf---
    \hypersetup{
       backref=true,                           % Permet d'ajouter des liens dans
       pagebackref=true,                       % les bibliographies
       hyperindex=true,                        % Ajoute des liens dans les index.
       colorlinks=true,  %Colorise les liens : true pour version numérique, false pour version d'impression
       breaklinks=true,                        % Permet le retour à la ligne dans les liens trop longs.
       urlcolor= blue,                         % Couleur des hyperliens.
       linkcolor= blue,                       % Couleur des liens internes.
       bookmarks=true,                         % Créé des signets pour Acrobat.
       %bookmarksopen=true,                    % Si les signets Acrobat sont créés,
                                               % les afficher complètement.
       pdftitle={Thème ENSG},                 % Titre du document.
                                               % Informations apparaissant dans
       pdfauthor={\author},                      % dans les informations du document
       pdfsubject={Feuille de style ENSG}           % sous Acrobat.
    }

%-----------------------



%-------------------------------------------------------------

\setcounter{tocdepth}{1} %profondeur de la table des matières

\title{Stage de fin d'étude ENSG \\ Estimation, analyse et prédiction de flux piétons \\ BigData et Machine Learning}

\bibliography{RAPPORT_ING3_PIERRAT_Jules}

%
%-------------------------------------------------------------
% Début du document
%--------------------------------------------------------------
\begin{document}
%--------------------------------------------------------------
\begin{titlepage}
%Inclusion des labels des entreprises
%Pour un seul label (à gauche), mettre NULL pour les 3e et 4e argument
\enterprise 
{logos/logo_ensg}
{\'Ecole Nationale des Sciences Géographiques}
{logos/logo_galigeo}
{Galigeo}

%Inclusion du titre

\maketitle{{\color{black}Rapport de stage}\\{\color{black}Cycle des Ingénieurs diplômés de l'ENSG 3\up{ème} année}}{logos/logo_ensg}



\infos{\author}{Septembre 2022}

\begin{center}
  \textit{Ne pas imprimer svp.}
  
  \textit{Préférez la lecture sur écran}
\end{center}

\end{titlepage}

%\begin{comment}
% ---Page du jury---
%---Page du jury---
%\newevenpage
\thispagestyle{plain}
\section*{Jury}
\vspace{0.5cm}

\textbf{Président de jury :} \\

Le président de jury

\vspace{0.5cm}

\textbf{Commanditaire :} \\

le commanditaire

\vspace{0.5cm}

\textbf{Encadrement de stage :} \\ 


qui a encadré ?

\vspace{0.5cm}

\textbf{Enseignant référent :} \\ 

qui a assuré le suivi pédagogique côté ENSG ?

\vspace{0.5cm}

\textbf{Rapporteur expert :} \\ 

qui est rapporteur du mémoire ?

\vspace{0.5cm}

\textbf{Responsable{\color{magenta}s} pédagogique{\color{magenta}s} du {\color{red} cycle Ingénieur} ou du {\color{magenta}MS$\circledR$ PPMD} :} \\



{\color{red}Serge Botton, IGN/ENSG/DPTS}

ou 

{\color{magenta}
Jacques Beilin, IGN/ENSG/DPTS

Jean-François Hangouët, IGN/ENSG/DIAS
}

\vspace{0.5cm}

\textbf{Gestion du stage :} \\ 

Anna Cristofol, IGN/ENSG/DSHEI

\vspace{0.5cm}


\copyright \hspace{0.3cm} ENSG

\section*{Stage de fin d'étude du xxx au xxx }
\vspace{0.3cm}
\textbf{Diffusion web :} $\boxtimes$ Internet \hspace{0.2cm}$\boxtimes$ Intranet Polytechnicum\hspace{0.2cm}
$\boxtimes$ Intranet ENSG\vspace{0.3cm}

\textbf{Situation du document :} 
\vspace{0.2cm}
\par
Rapport de stage de fin d'études présenté en fin de 3\up{ème} année du cycle des Ingénieurs
\vspace{0.3cm}


\newcounter{x}
\setcounter{x}{\getpagerefnumber{LastPage}-\getpagerefnumber{beginappendices}+1}

\textbf{Nombres de pages :} \getpagerefnumber{LastPage} pages dont \arabic{x} d'annexes
\vspace{0.3cm}

\textbf{Système hôte :} \LaTeX
\vspace{1cm}


\textbf{Modifications :} 
\begin{center}
\begin{tabular}{|c|c|c|>{\centering}p{6.5cm}|}
\hline 
EDITION & REVISION & DATE & PAGES MODIFIEES\tabularnewline
\hline
\hline 
1 & 0 & 09/2016 & Création\tabularnewline
\hline 

\end{tabular}
\end{center}
%------------------

%------------------------------------------------------------------------------
% Remerciements
\newevenpage
\chapter*{Remerciements}
\addcontentsline{toc}{chapter}{Remerciements}

Avant toute chose, je tiens à remercier le lecteur pour l’intérêt qu’il porte à mon rapport et j’espère qu’il trouvera ici tout ce pourquoi il est venu. Je veux remercier également les personnes m’ayant permis de réaliser dans les meilleurs conditions ce stage ainsi que celles ayant contribué à l’élaboration de ce rapport.

Tout d'abord, j'adresse mes remerciements à mon professeur, \textbf{Mr Loïc Landrieux de l’Ecole Nationale des Sciences Géographiques}, mon maître de stage qui m’a suivi tout au long de ce stage, m’a guidé et éclairé dans mes décisions.

Je tiens à remercier mon maître de stage, \textbf{Jean-Michel Gaudin, Product Leader chez Galigeo} pour son suivi et l’intérêt qu’il a porté à mes travaux réalisés pendant le stage. Je remercie également \textbf{Raimana Teina, Product Dev Leader chez Galigeo} qui m’a guidé dans mes travaux et grâce à qui j’ai énormément progressé et appris durant toute ma période de stage. Mes remerciements vont également à \textbf{Sébastien Connesson, COO de Galigeo}, qui m’a permis de comprendre au mieux l’organisation de l’entreprise, les relations internes, les enjeux et les rapports aux clients.

Je remercie évidement tout le reste de \textbf{l’équipe de Galigeo} pour son accueil, la confiance qu’ils m’ont accordée, leurs conseils et leur bienveillance. Je suis très heureux d’avoir pu travailler avec eux et me réjouis de continuer à le faire.

Enfin, après ces trois années fabuleuses je tiens à remercier toutes les personnes qui ont croisé mon chemin à \textbf{L’Ecole Nationale des Sciences Géographiques, mes professeurs, mes amis} et toutes les rencontres qui m’ont permis de grandir et de me préparer à cette nouvelle vie après les études.

Enfin, je tiens à remercier toutes les personnes qui m'ont conseillé et relu lors de la rédaction de ce rapport de stage : \textbf{ma famille, mon ami Antoine Rainaud} camarade de promotion.



%---Résumé (français)---
\begin{abstract}
\thispagestyle{empty}
	\vspace{1cm}

	Ce document a pour objectif de rapporter au mieux mon expérience de stage de fin d'études chez Galigeo, entreprise spécialisée dans le géomarketing.

  \paragraph{}

  En fin de cycle ingénieur à l'École Nationale des sciences géographiques, je dois réaliser un stage de 6 mois dans le milieu professionnel. Pour ma dernière année, je me suis spécialisé dans les techniques de systèmes d'informations (TSI) et j'ai pu découvrir en profondeur des notions comme le développement web, l'intelligence artificielle et la data science. Ce sont des domaines que j'ai adoré découvrir et je voulais en apprendre plus au sein d'une entreprise.

  Galigeo proposait un stage de machine learning en s'appuyant sur du big data ce qui correspondait exactement à mes envies. C'est une entreprise spécialisée en géomarketing ce qui est une notion toute nouvelle pour moi.

  \paragraph{}

  J'ai pu travailler sur des modèles de prédiction de flux piéton, de chiffre d'affaires, etc. J'ai eu la chance de réaliser de nombreux projets très intéressants et complets. Je détails donc mes travaux dans la suite de ce document.

	
	\vspace{1.5cm}
	
	\textbf{Mots clés :} Géomarketing, Machine Learning, Flux piéton
\end{abstract}
%-----------------------


%---Résumé (anglais)---
%\selectlanguage{english}
\begin{abstract}
\thispagestyle{empty}
	\vspace{1cm}
	
  This document aims to report as best as possible my experience as an intern at Galigeo, a company specialised in geomarketing.

  \paragraph{}

  At the end of my engineering studies at the National School of Geographic Sciences, I have to do a 6-month internship in a professional environment. For my last year, I specialised in computer science and I was able to discover in depth notions like web development, artificial intelligence and data science. These are areas I loved to discover and I wanted to learn more within a company.

  Galigeo offered an internship in machine learning based on big data, which was exactly what I wanted. It is a company specialised in geomarketing which is a very new concept for me.

  \paragraph{}

  I was able to work on pedestrian flow prediction models, turnover models, etc. I had the chance to realize many very interesting and complete projects. I will detail my work in the rest of this document.

	\vspace{1.5cm}
	
	\textbf{Key words:} Buisness Location Intelligence, Geomarketing, Pedestrian flow, Machine Learning

\end{abstract}
%----------------------

\selectlanguage{french}

%---Table des matières, des figures et des tableaux---
% \newevenpage
\tableofcontents



\newevenpage
\chapter*{Glossaire et sigles utiles}
\addcontentsline{toc}{chapter}{Glossaire et sigles utiles}

  \begin{acronym}
    \acro{BI}{Business Intelligence}
    \acro{CNN}{Convolutional Neural Network}
    \acro{CEO}{Chief Executive Officer}
    \acro{COO}{Chief Operating Officer}
    \acro{CRM}{Customer Relationship Management}
    \acro{DNN}{Deep Neural Network}
    \acro{ENSG}{\'Ecole Nationale des Sciences Géographiques}
    \acro{INSEE}{Institut national de la statistique et des études économiques}
    \acro{IRIS}{Îlots Regroupés pour l'Information Statistique}
    \acro{LSTM}{Long Short Term Memory}
    \acro{ML}{Machine Learning}
    \acro{POI}{Point of Interest}
    \acro{RetD}{Recherche et Développement}
    \acro{RNN}{Recurrent Neural Network}
    \acro{SaS}{Software as Service}
    \acro{SVM}{Support Vector Machine}
  \end{acronym}


%---Introduction------------------------------------------------------------------
% \newevenpage
\chapter*{Introduction}
  \addcontentsline{toc}{chapter}{Introduction}

Le géomarketing ou Location Business Intelligence en anglais est un pilier du marketing. Il étudie la variation des marchés dans l’espace. Les objectifs sont de modéliser offres et demandes en fonction de données économiques, sociales, culturelles, administratives et démographiques et leurs variations en fonction des géographies.

C’est un domaine essentiel pour les entreprises qui cherchent à développer leurs espaces d’action. En effet, c’est une solution qui aide à la prise de décision pour le développement d’un business. Il permet de choisir les sites stratégiques les plus appropriés pour implanter un nouveau commerce. La réalisation de modèle ou de simulation sont des outils essentiels en vue de comparer les atouts et risques d’une future implantation. En étudiant les espaces entourant toutes ses enseignes, une entreprise peut également anticiper la cannibalisation\footnote{Réduction du volume de vente d'une enseigne dû l'arrivée sur le marché d'une nouvelle enseigne de la même entreprise} ou la segmentation des portefeuilles en prenant tous les paramètres réunis en un point de l’espace.

Il permet également d’établir des stratégies de marketing rentables et efficaces en établissant des profils de susceptibles consommateurs. En modélisant de manière précise et orientée dans le sens des besoins de l’entreprise ces profils, on obtient alors une idée complète des déplacements et des comportements réels des consommateurs. Les stratégies de prospection et communication sont donc amenées à être plus efficaces.

La concurrence est également bien étudiée et cela permet de projeter la pérennité de l’entreprise dans le temps en alertant sur l’évolution des réseaux de concurrents.

Le géomarketing est la solution efficace afin d’appréhender parfaitement les territoires impactés par une nouvelle implantation et ainsi suivre son évolution tout au long de sa croissance.

\paragraph*{}

Ces dernières années, les solutions de géomarketing s’appuient de plus en plus sur des modèles de prédictions de plus en plus complexes et précis. La mise à disposition de modèles de flux piéton s’avère très utile pour permettre d’améliorer le géomarketing d’une compagnie. La demande concernant cette mesure est en augmentation et les entreprises vendeuses de solutions de géomarketing cherchent à obtenir les meilleurs modèles prédictifs pour répondre au mieux aux besoins de leurs clients.

Les algorithmes de Machine Learning sont des outils puissants pour estimer spatialement les flux piétons utiles aux analyses de géomarketing. Ils nécessitent cependant des quantités de données très importantes pour obtenir les précisions nécessaires.

Mon stage a consisté en partie à la réalisation de ce modèle prédictif. Ce genre de mission est typique au métier de Géo Data Scientist \footnote{Spécialiste de la donnée géographique} et c’est donc autour de cette mission que je développe mon rapport de stage.





%-------------------------------------------------------------------------------
%\end{comment}

\evenchapter[La nouvelle feuille de style \ensg]{La feuille\newline de style \ensg}

\textit{Voici la version 2 de la feuille de style \ensg. }

\section{Les fichiers}
\begin{itemize}
\item \texttt{themeensg.cls} : contient les personnalisations et macros utiles
\item \texttt{jury.tex} : pour la feuille de présentation du jury
\item le dossier \texttt{images} : il doit contenir toutes les images, il contient déjà le dossier logo avec celui de l'\ensg
\item \texttt{bibliographie.bib} contient la bibliographie
\end{itemize}

\section{Commandes personnalisées}

\begin{itemize}
\item \verb!\newevenpage! : identique à \verb!\newpage! mais en insère une page blanche de façon à débuter la nouvelle page sur un numéro de page impaire.
\item \verb!\evenchapter{titre}! : démarre un nouveau chapitre sur une page impaire,\\ \verb!\evenchapter[titre sommaire]{titre}! fonctionne aussi mais pas \verb!\evenchapter*{titre}!
\item idem pour \verb!\evenpart{titre}!
\end{itemize}

\section{Fichier source de cette doc}
Ce fichier \texttt{tex} contient toute la structure d'un rapport mais une bonne partie est désactivée car commentée par l'environnement \verb!\begin{comment} ... \end{comment}!

\subsection{Une sous-partie...}

On évitera si possible de faire des sous-sous-parties (\verb!\subsubsection!). Si vous en avez besoin, peut-être faut-il revoir la structures du document...

Au passage voici le code pour appeler une référence de la biblio \cite{globalpositioning} : \verb!\cite{globalpositioning}!

Pour plus d'infos sur la bibli, aller sur \url{http://bertrandmasson.free.fr/index.php?article27/}

\evenchapter[Le stage \ensg]{Le stage}

\section{Généralités}

\section{Prédiction de flux piéton}

\section{Prédiction de chiffre d'affaire}

\section{Autres missions}
\evenchapter[Bilan des travaux \ensg]{Bilan des travaux}

Aujourd'hui, tous les projets de




%\begin{comment}
%-------------------------------------------------------------------------------
% \newevenpage
\chapter*{Conclusion}
  \addcontentsline{toc}{part}{Conclusion}
  \vspace{1.5cm}


  Je suis très heureux du déroulement de ce stage. Les objectifs fixés et personnels ont été atteint et dépassé selon moi.

  \paragraph{}

  J'ai beaucoup progressé techniquement et j'ai pu approfondir les notions vues cette année qui me tenaient à coeur. J'ai découvert de nouveaux métiers, de nouveaux enjeux et de nouvelles méthodologies. J'ai pu travailler sur de nouveaux projets, tous différents et source de réflexion.

  J'ai également appris à travailler avec une équipe de développeurs, à tenir des objectifs, à répondre au besoin des clients. Je me suis rendu compte de l'importance de la compétitivité et de l'intérêt porté par toute sorte d'entreprise à des équipes performantes et modernes.

  Galigeo m'a très bien accueilli et j'ai adoré rencontrer et travaillé avec l'ensemble des collaborateurs. Je suis très heureux de leur proposition de CDI et j'ai donc décidé de continuer à travailler avec eux après ce stage.


%-------------------------------------------------------------------------------
% Insertion de la bibliographie
\newevenpage
%\printbibheading
\printbibliography[title={Bibliographie}]
\nocite{*}

\newevenpage
\listoffigures

\newevenpage
\listoftables
%----------------------------------------------------

\begin{appendices} 
\label{beginappendices}
\annexe[Planning du stage]{Planning du stage\newline Gannt}
\label{planning_gannt}

% Image of the offices
\begin{figure}[H]
  \centering
  \includegraphics[width=13cm]{images/graphs/gannt.png}
  \caption{Planning du stage}
\end{figure}

\annexe[Statistiques DNN]{Statistiques\newline Estimation de flux piétons}
\label{stat_dnn}

% Image of the offices
\begin{figure}[H]
  \centering
  \includegraphics[width=\linewidth]{images/graphs/stats_footfall.png}
  \caption{Différentes statistiques du modèle de DNN}
\end{figure}

\end{appendices} 
%\end{comment}
\end{document}